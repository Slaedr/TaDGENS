\documentclass[11pt]{article}

\usepackage[letterpaper, margin=2.5cm]{geometry}
\usepackage{amsmath}
\usepackage{amssymb}
\usepackage{graphicx}
\usepackage{caption}
\usepackage{subcaption}
\usepackage{multirow}
\usepackage{placeins}

\graphicspath{{../images/}}

\let\bld\boldsymbol

\bibliographystyle{plain}

\title{Discontinuous Galerkin method with Taylor basis functions \\
\large Proposal for MATH 579 project}
\author{Aditya Kashi}
\date{March 5, 2017}

\begin{document}

\maketitle

\section{Introduction}
We wish to investigate the solution of the compressible Euler equations in two-dimensions using Taylor basis functions in a discontinuous Galerkin (DG)) finite element method (FEM). Nodal Lagrange finite elements are by far the most commonly used as the polynomial basis of choice in fields like solid mechanics and fluid mechanics. However, an alternative exists in modal basis functions, such as Legendre basis and Taylor basis.

There are some advantages that the hierarchical, modal, Taylor basis functions have over nodal Lagrange basis functions \cite{luo_taylor, aizinger_scaleseparation}. 
\begin{itemize}
\item The set of basis functions of a certain polynomial degree contains the basis functions of all lower-degree sets of basis functions, ie., the basis is \emph{hierarchical}. This makes easier the implementation of p-adaptation (dynamically changing polynomial degree based on accuracy requirements during the simulation) and p-multigrid solvers (utilizing corrections to the solution computed from lower-order solves).
\item The Taylor basis expansion remains same and the basis functions retain the same form irrespective of the geometric type of elements. This makes for easier implementation of codes that work both on triangles and quadrangles, or both tetrahedra, prims and hexahedra. Another aspect of this is that the number of degrees of freedom in case of Taylor basis is smaller than that in case of nodal basis for non-simplicial elements, while maintaining optimal order of accuracy.
\item Thirdly, they make implementation of reconstruction DG methods (described later) efficient and elegant. This is because the spatial derivatives of the unknowns are readily available as those are the degrees of freedom. Using them, higher derivatives can be reconstructed to improve the accuracy of the scheme, as done in high-order (higher than 1st order) finite volume methods.
\end{itemize}

\section{Governing equations}

The compressible Euler equations, in $n_d$ spatial dimensions, are
\begin{equation}
\frac{\partial \bld{u}}{\partial t} + \sum_{j=1}^{n_d} \frac{\partial \bld{F}_j}{\partial x_j} = \bld{0} \quad \bld{x} \in \Omega, t \in [0,T]
\label{conservativeGE}
\end{equation}
with some initial condition
\begin{equation}
\bld{u}(\bld{x},0) = \bld{u}_0(\bld{x})
\end{equation}
and some combination of several possible types of boundary conditions.
$\bld{u}(\bld{x},t)$ is the vector of conserved variables, and $\bld{F}(\bld{u}(\bld{x},t))$ are the flux functions. For two-dimensional flows, these are
\begin{equation}
\bld{u} = 
\begin{bmatrix}
\rho \\ \rho u_1 \\ \rho u_2 \\ \rho e
\end{bmatrix}, \quad
\bld{F}_1 = 
\begin{bmatrix}
\rho u_1 \\ \rho u_1^2 + p \\ \rho u_1 u_2 \\ u_1 (\rho e + p)
\end{bmatrix}, \,
\bld{F}_2 = 
\begin{bmatrix}
\rho u_2 \\ \rho u_1 u_2 \\ \rho u_2^2 \\ u_2 (\rho e + p)
\end{bmatrix}.
\end{equation}
In the above, $\rho$ is the fluid density, $u_j$ are the velocity components, $e$ is the specific energy, and $p$ is the pressure. The governing equations are closed by the ideal gas equation of state
\begin{equation}
p = \rho (e - \frac12 \vert \bld{u} \vert^2).
\end{equation}

A weak formulation is derived in a `broken' Sobolev space and discretized by finite element method \cite{luo_taylor}.

\section{Taylor basis functions}
In a Taylor basis, a function is expressed as a Taylor expansion about the element center. In 2D, a quadratic or P2 expansion in an element e would be written as
\begin{equation}
u_h = u_c + \frac{\partial u}{\partial x}\vert_c(x-x_c) + \frac{\partial u}{\partial y}\vert_c(y-y_c) + \frac{\partial^2 u}{\partial x^2}\vert_c \frac{(x-x_c)^2}{2} + \frac{\partial^2 u}{\partial y^2}\vert_c \frac{(y-y_c)^2}{2} + \frac{\partial^2 u}{\partial x\partial y}\vert_c (x-x_x)(y-y_x)
\label{eqn:taylorexpn_orig}
\end{equation}
where the subscript 'c' indicates the quantity at the element's geometric center. Let $A_e$ be the element area. Then we can take an average of both sides over the element to get
\begin{multline}
\tilde{u} = u_c + \frac{\partial u}{\partial x}\vert_c \frac{1}{A_e}\int_{\Omega_e}(x-x_c)d\mu + \frac{\partial u}{\partial y}\vert_c \frac{1}{A_e}\int_{\Omega_e} (y-y_c)d\mu \\ \, + \frac{\partial^2 u}{\partial x^2}\vert_c \frac{1}{A_e}\int_{\Omega_e} \frac{(x-x_c)^2}{2}d\mu + \frac{\partial^2 u}{\partial y^2}\vert_c \frac{1}{A_e}\int_{\Omega_e} \frac{(y-y_c)^2}{2}d\mu + \frac{\partial^2 u}{\partial x\partial y}\vert_c \frac{1}{A_e}\int_{\Omega_e} (x-x_x)(y-y_x)d\mu \\
= u_c + \frac{\partial^2 u}{\partial x^2}\vert_c \frac{1}{A_e}\int_{\Omega_e} \frac{(x-x_c)^2}{2}d\mu + \frac{\partial^2 u}{\partial y^2}\vert_c \frac{1}{A_e}\int_{\Omega_e} \frac{(y-y_c)^2}{2}d\mu + \frac{\partial^2 u}{\partial x\partial y}\vert_c \frac{1}{A_e}\int_{\Omega_e} (x-x_x)(y-y_x)d\mu
\end{multline}
where $\tilde{u}$ is the average value of the unknown function over the element, and $\mu$ is the area measure in 2D. This last equation can be subtracted from \eqref{eqn:taylorexpn_orig} to get
\begin{multline}
u_h = \tilde{u} + \frac{\partial u}{\partial x}\vert_c(x-x_c) + \frac{\partial u}{\partial y}\vert_c(y-y_c) + \frac{\partial^2 u}{\partial x^2}\vert_c \left( \frac{(x-x_c)^2}{2} - \frac{1}{A_e}\int_{\Omega_e} \frac{(x-x_c)^2}{2}d\mu \right) \\ + \frac{\partial^2 u}{\partial y^2}\vert_c \left( \frac{(y-y_c)^2}{2} -\frac{1}{A_e}\int_{\Omega_e} \frac{(y-y_c)^2}{2}d\mu \right) + \frac{\partial^2 u}{\partial x\partial y}\vert_c \left( (x-x_x)(y-y_x) - \frac{1}{A_e}\int_{\Omega_e} (x-x_x)(y-y_x)d\mu \right).
\label{eqn:taylorexpn}
\end{multline}
This gives us our basis functions and corresponding degrees of freedom on a physical element $\Omega_e$. We also normalize the basis functions to get a better-conditioned discrete problem. Finally, the basis functions used for a P2 Taylor finite element are
\begin{align}
&B_1(\bld{x}) = 1 \\
&B_2(\bld{x}) = \frac{(x-x_c)}{\Delta x} \\
&B_3(\bld{x}) = \frac{(y-y_c)}{\Delta y} \\
&B_4(\bld{x}) = \left( \frac{(x-x_c)^2}{2\Delta x^2} - \frac{1}{A_e}\int_{\Omega_e} \frac{(x-x_c)^2}{2}d\mu \right) \\
&B_5(\bld{x}) = \left( \frac{(y-y_c)^2}{2\Delta y^2} -\frac{1}{A_e}\int_{\Omega_e} \frac{(y-y_c)^2}{2}d\mu \right) \\
&B_6(\bld{x}) = \left( \frac{(x-x_x)(y-y_x)}{\Delta x\Delta y} - \frac{1}{A_e}\int_{\Omega_e} (x-x_x)(y-y_x)d\mu \right)
\end{align}
with the corresponding coefficients, the degrees of freedom, being
\begin{align}
\tilde{U} := \tilde{u} \\
U_x := \frac{\partial u}{\partial x}\vert_c \Delta x \\
U_y := \frac{\partial u}{\partial y}\vert_c \Delta y \\
U_{xx} := \frac{\partial^2 u}{\partial x^2}\vert_c 2\Delta x^2 \\
U_{yy} := \frac{\partial^2 u}{\partial y^2}\vert_c 2\Delta y^2 \\
U_{xy} := \frac{\partial^2 u}{\partial x\partial y}\vert_c \Delta x\Delta y
\end{align}
where $\Delta x$ and $\Delta y$ are $\frac12 (x_{max}-x_{min})$ and $\frac12 (y_{max}-y_{min})$ respectively, and subscripts max and min indicate maximum and minimum values over the element. Note that symbols such as $U_x$ include their respective normalization factors.

This can be similarly extended to any desired polynomial degree in any number of dimensions.

\section{Reconstruction DG scheme}
One issue with the DG FEM method is its high cost compared to finite volume schemes for low to intermediate levels of desired accuracy. They become competitive only when a high-enough level of accuracy is required. DG FEM is also more expensive than a continuous FEM scheme of the same order on the same grid, since degrees of freedom are not shared among elements. One way of reducing the cost is the reconstruction DG (RDG) FEM scheme \cite{luo_rdg}. A Taylor basis RDG scheme can be implemented in a reasonably straightforward manner to give high-order accuracy using less degrees of freedom than a DG scheme.

Consider a DG P1 scheme with Taylor elements. The degrees of freedom are the mean value over the element and the first spatial derivatives at the element centre. Using this data, the three second spatial derivatives can be reconstructed using data from face-neighboring elements, assuming a smooth enough solution in the region. Consider the union of a given element $\Omega_i$ and its face-neighboring elements $\Omega_j, j = 1,2,3$ or $j = 1,2,3,4$. We write the unknown $u(x)$ as a Taylor series on this union (using the nomenclature introduced in the previous section).
\begin{equation}
u(\bld{x}) = \tilde{U}_i + U_{xi}B_{i_2}(\bld{x}) + U_{yi}B_{i_3}(\bld{x}) + U_{xxi} B_{i_4}(\bld{x}) + U_{yyi} B_{i_5}(\bld{x}) + U_{xyi} B_{i_6}(\bld{x})
\end{equation}
where $B$ are the basis functions on element $i$ and the derivatives are those at the centre of element $i$. We can also Taylor-expand the first derivatives.
\begin{align}
\frac{\partial u}{\partial x} &= U_{xi}/\Delta x_i + U_{xxi} B_2 / \Delta x_i + U_{xyi}/\Delta x_i \\
\frac{\partial u}{\partial y} &= U_{yi}/\Delta y_i + U_{xyi} B_2 / \Delta y_i + U_{yyi}/\Delta y_i
\end{align}

Note that we already know the first-order derivatives as those are our degrees of freedom. The second derivatives are the unknowns we wish to reconstruct. To this end, the above functions can be evaluated at the element-centres $\bld{x}_j$ of each of the face-neighboring elements $j$.
\begin{align}
u_j &= \tilde{U}_i + U_{xi}B_{i_2}(\bld{x}_j) + U_{yi}B_{i_3}(\bld{x}_j) + U_{xxi} B_{i_4}(\bld{x}_j) + U_{yyi} B_{i_5}(\bld{x}_j) + U_{xyi} B_{i_6}(\bld{x}_j) \\
\frac{\partial u}{\partial x}\big|_j &= U_{xi}/\Delta x_i + U_{xxi} B_{i_2}(\bld{x}_j) / \Delta x_i + U_{xyi} B_{i_3}(\bld{x}_j) /\Delta x_i \\
\frac{\partial u}{\partial y}\big|_j &= U_{yi}/\Delta y_i + U_{xyi} B_{i_2}(\bld{x}_j) / \Delta y_i + U_{yyi}B_{i_3}(\bld{x}_j)/\Delta y_i
\end{align}
This can be expressed as the following linear system.
\begin{equation}
\begin{bmatrix}
B_4^j & B_5^j & B_6^j \\
B_2^j & 0 & B_3^j \\
0 & B_3^j & B_2^j
\end{bmatrix}
\begin{bmatrix}
U_{xxi} \\ U_{yyi} \\ U_{xyi}
\end{bmatrix} =
\begin{bmatrix}
u_j - (\tilde{U}_i + U_{xi}B_2^j + U_{yi}B_3^j) \\
\frac{\Delta x_i}{\Delta x_j}U_{xj} - U_{xi} \\
\frac{\Delta y_i}{\Delta y_j}U_{yj} - U_{yi}
\end{bmatrix} =:
\begin{bmatrix}
R_1^j \\ R_2^j \\ R_3^j
\end{bmatrix}
\end{equation}
where the superscript $j$ indicates that the function is evaluated at the element-centre of $j$. Writing these equations for all face-neighbors $j$, we get a 9x3 system for a triangle and a 12x3 system for a quadrilateral, in the three variables - the three second derivatives at the centre of cell $i$. This can be solved in a least squares sense to give a unique solution. Further, a WENO reconstruction can be used to suppress spurious oscillations to preserve nonlinear stability \cite{luo_hweno}.

\section{Outline of the project}
One can think of the following questions that need answering; we have not found them in papers yet. They may be trivial, however.
\subsection{Taylor basis DG}
\begin{itemize}
\item Show that the degrees of freedom described are unisolvent, and thus that a `Taylor element' is a valid finite element.
\end{itemize}
\subsection{RDG method}
\begin{itemize}
\item Prove that the least-squares system will have a solution, ie., that the 9x3 or 12x3 matrix has full column rank.
\end{itemize}

On the implementation side, the tentative plan is as follows. First implement Taylor basis DG FEM for the compressible Euler equations on 2D hybrid (triangles and quadrangles) grids. Total variation diminishing Runge-Kutta explicit time-stepping will be used.

Two test cases are planned - the isentropic vortex problem on a periodic domain, for which an analytical solution is available, and subsonic inviscid flow over a cylinder, for which error can computed by measuring the entropy generated. Note that entropy generation should be zero for an ideal inviscid subsonic flow. Also, curved elements are necessary for this case \cite{bassi_dgeuler}.

Once that is accomplished, tentatively, the reconstruction DG scheme will be implemented to investigate the order of accuracy achieved and the computing resources needed as compared to a regular DG method of same formal order of accuracy.

\bibliography{refs}
\end{document}